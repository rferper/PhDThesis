%%%%%%%%%%%%%%%%%%%%%%%%%%%%%%%%%%%%%%%%%%%%%%%%%%%%%%%%%%%%%%%%%%%%%%%%%%%%%%%%%%%%%%
%%																					%%
%%								CONCLUSIONS AND FUTURE WORK							%%	
%%																					%%
%%%%%%%%%%%%%%%%%%%%%%%%%%%%%%%%%%%%%%%%%%%%%%%%%%%%%%%%%%%%%%%%%%%%%%%%%%%%%%%%%%%%%%

% Intro comment
In this monograph, we have primarily focused on solving some open problems related to the study of characterizations and intersections of classes of fuzzy implication functions. Specifically, we have addressed four well-defined objectives which have been achieved satisfactorily. Let us point out that at the end of each core chapter a detailed discussion regarding the conclusions and future work linked to each objective have been provided, so in this final chapter our aim is to give a concise overview of what has been achieved and to comment on the future outlook that can be derived from the problems studied in this thesis.

% Summary Chapter 3
To start with, in Chapter \ref{chapter:heimplications} we have provided an axiomatic characterization of generalized $(h,e)$-implications. This result has been obtained by first presenting a representation theorem for these operators which discloses that its structure is determined by the horizontal threshold construction method applied to two subfamilies of some new classes of fuzzy implication functions, called $(f,g)$ and $(g,f)$-implications, which can be viewed as generalizations of Yager's implications. We have thoroughly studied the additional properties of these two new classes, characterizing the two subfamilies that are related to generalized $(h,e)$-implications, called $(f,e)$ and $(g,e)$-implications. Thus, the characterization of generalized $(h,e)$-implications is obtained by reinterpreting the representation theorem taking into account the key properties in the axiomatic characterization of $(f,e)$ and $(g,e)$-implications. The three characterizations presented in this chapter rely on two novel properties, \LIex and \LIey, which are modifications of the law of importation.

% Summary Chapter 4
Second, in Chapter \ref{chapter:tpowerinvariant} we have done an exhaustive study on the $T$-power invariance property, denoted by \PIT, where $T$ is a continuous Archimedean t-norm. Particularly, we have presented two new families of fuzzy implication functions characterized by the fact that they are invariant with respect to the positive powers of a strict/nilpotent t-norm $T$, denoted by strict/nilpotent $T$-power invariant implications. We have deeply investigated when the members of these families fulfill other additional properties apart from the invariance. In view of the corresponding results, we have exposed that, although \PIT is too restrictive in combination with properties like \LI or \NP, for others such as \EP or \TC we obtain interesting solutions in the form of parametric subfamilies whose expression depends on an additive generator of the corresponding t-norm. Further, the detailed study of the additional properties has allowed us to determine the intersection of strict/nilpotent $T$-power invariant implications with other ten well-known families of fuzzy implication functions.  Since we have disclosed that these two new families have almost no intersection with the others, we can conclude that their structure stands out with respect to the most well-known families.

% Summary Chapter 5
Third, in Chapter \ref{chapter:chsnimplications} we have presented considerable advances on the open problem of the characterization of $(S,N)$-implications when $N$ is a non-continuous fuzzy negation. To begin with, we have provided a first general result that proves the equivalence between this problem and the determination of the completions of t-conorms which are unknown in certain subregions of the unit square that depend on the discontinuities of the fuzzy negation. Next, by the duality between t-norms and t-conorms, we have focused on the problem of finding the completions of pre-t-norms defined on the corresponding regions. In view of the complexity of the resulting problem, it has been convenient to restrict the study to the eight regions derived from the particular case when $N$ has only one point of discontinuity and $S$ is a continuous t-conorm. Regarding this problem, we have characterized all the continuous completions of cancellative/conditional cancellative pre-t-norms defined on the eight regions of interest and in each case we have provided the corresponding constructions in terms of an additive generator. From the obtained results, we have presented a second characterization of $(S,N)$-implications based on the explicit construction of the t-conorm $S$ in the particular case when $N$ has one point of discontinuity and $S$ is the maximum or a continuous Archimedean t-conorm.

% Summary Chapter 6
Finally, in Chapter \ref{chapter:sd} we have taken a more applied approach and we have introduced a novel framework for subgroup discovery based on the use of fuzzy implication functions to model subgroups as fuzzy rules. In the first part of this chapter, we have focused on setting the basis of our new perspective by generalizing several quality measures and studying which additional properties should be satisfied by the fuzzy operators involved. At the core of this discussion, we have introduced a new additional property of fuzzy implication functions which is related to the generalized modus ponens, denoted by \MTC. This property has been a key point both for an adequate behavior of the defined measures and the definition of an optimistic estimate for a faster optimization of the fuzzy weighted relative accuracy. Next, we have designed and implemented some subgroup discovery algorithms, which have been tested in different real datasets. From the experimental results we conclude that our new perspective has a lot of potential since it provides valuable knowledge which is different from existing approaches. Particularly, our algorithms stand out with respect to other perspectives because the introduction of fuzzy implication functions allows the modeling of a numeric target variable as a linguistic variable and subgroups as IF-THEN rules, which are interpreted as logical conditionals rather than the co-occurrence of the antecedent and the consequent.

% Summary Relations
Even though the standpoint of this thesis has been to deal with the specific objectives mentioned above, we would like to point out some unexpected and interesting connections between our results and other problems:
\begin{itemize}
	\item One of the solutions of \EP for strict $T$-power invariant implications has resulted to be the preference implication, which was independently introduced in \cite{Baczynski2020B} as the solution of the four distributivities with respect to the operators of the pliant system. Thus, this operator has been disclosed from two very different perspectives and, by putting the two studies together we obtain a fuzzy implication function which satisfies \PIT, \EP and the four distributivities (see Remark \ref{remark:preferenceimplication}).
	\item The intersections between $(U,N)$-implications and strict/nilpotent $T$-power invariant implications have been characterized with the exception of a specific fuzzy implication function, which has turned out to be related to uninorms not locally internal on the boundary (see Remark \ref{remark:strict:CommentsIntUN}). The main classes of uninorms are locally internal on the boundary, and the study of uninorms which do not satisfy this condition is an active research area \cite{Xie2022}.
	\item Similarly to the point above, the intersection of $RU$-implications and nilpotent $T$-power invariant implications have been also characterized except for one specific case (see Remark \ref{remark:RU}). Particularly, this problem leads to determine if a certain fuzzy implication function is an $RU$-implication or not. Even if we have not answered this question, we have exposed that it is related to the generalizations of $RU$-implications in which $U$ is a commutative semi-uninorm \cite{Krol2011,Ouyang2012}.
	\item The equivalence between the study of the completions of t-norms and the characterization of $(S,N)$-implications has been the most relevant connection between two different lines of study disclosed in this thesis.  Indeed, the completion problem is a classical problem on the study of t-norms \cite{Kimberling1973,Darsow1983,Bezivin1993}, for which we have unveiled another motivation. 
\end{itemize}
These examples support the understanding of the study of fuzzy operators as a whole, in which results from apparently independent problems may be useful for a concrete goal.

% Outlook
Although in this monograph we have tried to close all the proposed objectives as much as possible, few open problems have remained and new ones have arisen. We have distinguished between four different motivations for future studies:
\begin{itemize}

\item \textit{To further study the new additional properties introduced in this thesis.}  Throughout this monograph various new additional properties of fuzzy implication functions have been introduced in different contexts. It would be convenient to study in more detail these properties for various reasons:
\begin{itemize}
	\item A deeper study of the properties \LIex and \LIey may help to complete the study of the independence between the properties in the characterization of $(f,e)$-implications with $f(0)=+\infty$ (see Table \ref{independence} and the ensuing discussion). Moreover, it would be interesting to contextualize these generalizations of the law of importation with others \cite{Baczynski2020,Massanet2011B}.
	\item Analogously to the point above, we should study in more detail the properties {\bf (R1)} and {\bf (R2)} introduced for the characterization of $(S,N)$-implications. Not only for ensuring the independence between the properties in the corresponding result, but also to see if these conditions can be simplified. 
	\item The property \MTC has been introduced for ensuring the desired behavior of fuzzy implication functions and t-norms used to model subgroups as fuzzy rules. Since in Chapter \ref{chapter:sd} we have focused on a practical application, we have not thoroughly studied this property from a theoretical point of view. Indeed, we have limited ourselves to provide various examples satisfying this property derived from existing families of fuzzy implication functions. Therefore, a more detailed study of this property could be made to better contextualize its relevance in the literature related to the study of additional properties of fuzzy implication functions. On the other hand, since this property is strictly related to \TC, it would be even more interesting to study the adequacy and meaning of considering \MTC in other application areas in which the generalized modus ponens as inference mechanism is considered.
\end{itemize}

\item \textit{To address some unsolved problems related to our objectives.} Apart from providing the independence between the properties in the characterizations of $(f,e)$ and $(S,N)$-implications just mentioned above, other two open problems can be highlighted:
\begin{itemize}
	\item To complete the characterization of the intersections between strict/nilpotent $T$-power invariant implications, $(U,N)$ and $RU$-implications. As we have already mentioned in this chapter, these two problems are related to two very specific cases connected with the study of uninorms which do not belong to the most usual classes.
	\item Although we have studied in detail the completion problem linked to the characterization of $(S,N)$-implication functions and significant advances have been presented, quite future work is needed to completely solve the problem. Particularly, the general situation of the completion of t-conorms unknown in the region $(\Ran N \times [0,1]) \cup (([0,1] \setminus \Ran N) \times \Ran N)$ has to be addressed.  This means that for obtaining a characterization of $(S,N)$-implications based on the construction of $S$ without further restrictions, ideally we would have to resolve the completion problem when $N$ has a numerable number of discontinuities and $S$ is non-continuous. In view of the complexity of the results presented in this monograph for the particular case when $N$ has only one point of discontinuity and $S$ is a continuous t-conorm, this objective is clearly very difficult to achieve. Nonetheless, in Sections \ref{subsection:sn:promisingresult} and \ref{section:sn:conclusions} concrete guidelines on this open problem are given in the case when $S$ is a continuous t-conorm. Specifically, we have exposed that the results provided in Chapter \ref{chapter:chsnimplications} are stronger than they may appear at first glance, and they form a solid basis for future studies.
\end{itemize}

\item \textit{To deepen in the study of characterizations and intersections of the families considered in this thesis:}
\begin{itemize}
	\item In this monograph, the additional properties of $(f,g)$ and $(g,f)$-implications have been studied, but we have only provided a characterization for the subfamilies of $(f,e)$ and $(g,e)$-implications. Therefore, the problem of characterizing $(f,g)$ and $(g,f)$-implications remains open.  Further, in view of the number of classes defined as generalizations of Yager's implications (see Table \ref{table:families_FI}), it would be interesting to study the intersections between the two considered families and other generalizations.
	\item As the generalization of strict/nilpotent $T$-power invariant implications, the characterization and study of fuzzy implication functions which are invariant with respect to the positive powers of continuous non-Archimedean t-norms could be studied.
\end{itemize}

\item \textit{To delve into the potential of fuzzy implication functions in applications:}
\begin{itemize}
	\item In this monograph we have presented subgroup discovery as a new application area for fuzzy implication functions, in which the incorporation of these operators has proven to be beneficial for a variety of reasons. Nonetheless, Chapter \ref{chapter:sd} has been a first approach to this topic, so there are numerous avenues for further development in the future. Focusing on the study of fuzzy operators, it would be interesting to study the meaning behind obtaining different best subgroups when we change the fuzzy implication function used for modeling the logical conditional. More generally, future work on this problem could be related to the incorporation of search heuristics, the improvement of the code, the proposal of other quality measures or the generalization of the description language.
	\item In Chapter \ref{chapter:tpowerinvariant} we have studied the families of strict/nilpotent $T$-power invariant implications as two families characterized by the fact that they satisfy \PIT, a property that was introduced in \cite{Massanet2017} for its potential applications in approximate reasoning when fuzzy hedges modeled as powers of continuous t-norms are considered. Nonetheless, up to our knowledge, the applicability of this additional property for a particular problem has not been tested. Thus, it would be interesting to consider the two families studied in this monograph in a practical application. It is worth mentioning that, although the possibility of considering strict/nilpotent $T$-power invariant implications for our subgroup discovery algorithms has been contemplated, it has not been feasible due to the lack of meaningful solutions when the newly introduced property \MTC and \PIT are jointly imposed in the case when $T$ is a continuous Archimedean t-norm. To further explore this situation, we could study whether considering the more general situation in which $T$ is a non-Archimedean continuous t-norm leads to more interesting solutions for this conjunction of required properties.
\end{itemize}
\end{itemize}