%%%%%%%%%%%%%%%%%%%%%%%%%%%%%%%%%%%%%%%%%%%%%%%%%%%%%%%%%%%%%%%%%%%%%%
%%																	%%
%%								INTRODUCTION						%%	
%%																	%%
%%%%%%%%%%%%%%%%%%%%%%%%%%%%%%%%%%%%%%%%%%%%%%%%%%%%%%%%%%%%%%%%%%%%%%

% The end of determinism
Historically, sciences had a strict deterministic view of reality in which they assumed that every event can be completely explained by its causes. For this school of thought, any uncertainty is caused by human ignorance about what is already predetermined. For instance, Jacob Bernoulli (1654-1705) or Pierre-Simon Laplace (1749-1827), widely known for their contributions to probability theory \cite{Bernoulli1713,Laplace1812}, shared a deterministic view of the world. In the words of Pierre-Simon Laplace \cite{Laplace1902}: ``Given for one instant an intelligence which could comprehend all the forces by which nature is animated and the respective situation of the beings who compose it – an intelligence sufficiently vast to submit these data to analysis – it would embrace in the same formula the movements of the greatest bodies of the universe and those of the lightest atom; for it, nothing would be uncertain and the future, as the past, would be present to its eyes''. At the end of the 19th century and the beginning of the 20th century, several breakthroughs questioned the adequacy of strict determinism as a model of reality: Gregor Mendel's (1822-1884) experimental findings on inheritance that caused the inception of the modern age of genetics more than three decades later \cite{Gayon2016}; Ludwig Boltzmann's (1844-1906) statistical interpretation of thermodynamics \cite{Swendsen2010}; the Brownian motion empirically discovered by Robert Brown (1773-1858) and lately modeled by Louis Bachelier (1870-1946) \cite{Seth2020}; Bertrand Russell's (1872-1970) set-theoretical paradox \cite{Godehard2004}; Werner Heisenberg's (1901-1976) uncertainty principle in quantum mechanics \cite{Cassidy1992}; or Kurt Gödel's (1906-1978) incompleteness theorems \cite{Raatikainen2022}. These advances were a turning point in science history and they gave rise to a more realistic worldview in which dealing with uncertainty was the mainstay \cite{Raper2020}.

% The emergence of multivaluated logics
This new current also reached the field of mathematical logic and philosophy, in which interpreting and discussing terms like ``vagueness'' caught the attention of many scholars \cite{Russell1923,Black1937,Hempel1939}. At this point, the truth-bivalence of classical logic aroused several controversies which prompted the study of other alternatives. In \cite{Russell1923} Bertrand Russell argued that the law of excluded middle does not hold when symbols are vague and concluded that vagueness is precisely one degree of truth; Jan Łukasiewicz (1878-1956) proposed a new logic with three degrees of truth, ``true'', ``false'' and ``unknown'' \cite{Lukasiewicz1920}; Emil Leon Post (1897-1954) introduced the idea of additional truth degrees \cite{Post1921}; and Luitzen Egbertus Jan Brouwer (1881-1996) introduced intuitionistic logic as a mathematical logic where the law of excluded middle was not imposed, for which it was proved in 1932 by Gödel that it has no interpretation as a finite-valued logic \cite{Godel1932}. On the basis of these pioneering works, the branch of multi-valued logics had a thorough development in subsequent years, both in theory and applications \cite{Gottwald2001}.

% The introduction of fuzzy logic and contributions of Lofti A. Zadeh
Within this scenario, Lofti A. Zadeh (1921-2017) expressed in 1962 the necessity of a new framework for processing uncertainty \cite{Zadeh1962}:  ``...we need a radically different kind of mathematics, the mathematics of fuzzy or cloudy quantities which are not describable in terms of probability distributions.''. Three years later, he published his famous seminar paper ``Fuzzy Sets'', in which he presented a new type of sets characterized by the fact that they do not have precise boundaries and for which membership is a matter of degree \cite{Zadeh1965}. From this generalization of classical sets, he proposed a mathematical method with which processing information based on natural language descriptions like ``The temperature is high'' or ``Alex is tall'' became possible. In posterior papers \cite{Klir1996}, Zadeh developed the theory of fuzzy logic as a more adequate formalism to handle the imprecision of human reasoning \cite{Dubois2016}.

% The boom of fuzzy logic
The term ``fuzzy logic'' can be understood from two different points of view, the narrow and the wider sense \cite{Marks1994}. In the narrow sense, fuzzy logic is a multi-valued logic in which truth degrees lie within the real interval $[0,1]$, where $0$ indicates ``absolute falsity'' and 1 indicates ``absolute truth''. However, in the wider sense fuzzy logic is almost synonymous with the theory of fuzzy sets. Although fuzzy logic can be systematically studied as a multi-valued mathematical logic \cite{Hajek1998}, the utmost motive of Zadeh's ideas were to use fuzzy logic as a theory of approximate reasoning whereby truth degrees act as modifiers of the fuzzy statements they apply to. This novel theory began to flourish in industrial applications in the early 1970s, particularly in the field of expert systems \cite{Gaines1985}. Further on, in the 1980s it started the period called the ``fuzzy boom'' due to the large number of fuzzy logic based products that emerged, especially in Japan \cite{Seising2009,Garrido2012}. Indeed, some examples in which fuzzy technology was incorporated are: household appliances such as washing machines, thermostats, cameras, rice cookers, microwave ovens, air conditioners...; the Hitachi subway system in Sendai installed in 1985; vehicle's auto transmission and antiskid braking systems \cite{VonAltrock1994,Ivanov2015}; among many others \cite{Dutta1993}.

% The controversies
Although fuzzy logic had an enthusiastic reception in the East, in the West this new theory was not lacking in criticism. According to several scholars, fuzzy logic was ``content-free'' or ``probability in disguise'', pointing out that for them fuzzy logic had nothing interesting or new to offer \cite{Zadeh1996}. Zadeh did not sharpen this confrontation, since he considered that probability and fuzzy logic were complementary rather than rivals \cite{Zadeh1995}. It has been widely discussed now that fuzzy degrees of truth are not the same as probability percentages \cite{Kosko1990}. In brief, probabilities measure whether something will occur or not, and fuzziness measures the degree to which something occurs or some condition exists. For instance, the statement ``There is a 30 percent of surviving this surgery'' conveys the probability of living or dying, but ``The surgery causes 30 percent harm'' means that the process can cause harm to some extent. For a more detailed example of the difference between membership degrees and probabilities, we refer the reader to \cite{Bezdek2013}. Nowadays, fuzzy logic is a well-established discipline with several theoretical ramifications and a wide variety of contemporary application areas: computing with words \cite{Gupta2022}, fuzzy control \cite{Precup2011}, decision making \cite{Liu2020,Blanco-Mesa2017,Mardani2015}, image processing \cite{Bloch2023}, data mining and machine learning \cite{Mirzakhanov2020}, neural networks \cite{Dombi2021}, genetic algorithms \cite{Herrera2008}, knowledge discovery \cite{Ropero2011,Herrera2011,Papageorgiou2013}, medicine \cite{Uma2022}, robotics \cite{Mac2016}... 

% Fuzzy operators
One of the most important branches of fuzzy logic corresponds to the study of fuzzy operators, which are used to operate between membership values or truth degrees. Traditionally, many fuzzy concepts were defined as a generalization of the corresponding one in classical logic. Following this reasoning, the main classical logic connectives have been generalized: the intersection or conjunction is defined as a fuzzy conjunction (usually a t-norm); the union or disjunction is defined as a fuzzy disjunction (usually a t-conorm); the negation or the complement is defined as a fuzzy negation; and the conditionals are represented by fuzzy implication functions. However, the study of fuzzy operators goes beyond logic connectives and it intersects with the study of aggregation functions. Aggregation functions (also called aggregation operators) are used for combining and merging values into a single one according to a certain objective. Since fuzzy operators play an important role in a wide variety of applications, many different types have been defined. To illustrate this fact we refer the reader to some books exclusively devoted to this topic \cite{Klement2000,Calvo2002,Baczynski2008,Beliakov2010,Grabisch2009,Alsina2006}. Although other domains besides $[0,1]$ have been considered in the literature \cite{Goguen1967,Munar2023}, typically fuzzy operators are defined as functions $F:[0,1]^n \to [0,1]$ that fulfill some set of conditions (monotonicity, continuity, associativity, commutativity, boundary conditions...). However, these conditions are usually general enough to allow the existence of many different operators of a certain kind. This results in the more specific study of different classes of operators that fulfill a certain set of conditions, in which desired additional properties apart from the ones in the operator's definition can be included. Thus, from a more theoretical point of view, the study of fuzzy operators falls within the scope of functional equations \cite{Kuczma1968,Aczel1966}. This monograph is mainly devoted to the study of fuzzy implication functions from this latter perspective.

% ----- Fuzzy Implication Functions ----- %
Fuzzy implication functions are defined as functions $I:[0,1]^2 \to [0,1]$ which are decreasing with respect to the first variable, increasing with respect to the second variable and they coincide with the classical implication in $\{0,1\}^2$ \cite{Baczynski2008,Fodor1994}. In the same way boolean implications are employed in inference schemas like modus ponens, modus tollens, etc., fuzzy implication functions play a similar role in the generalization of these schemas modeling the corresponding conditionals which are called fuzzy IF-THEN rules. These rules are widely used in approximate reasoning, wherein from imprecise inputs and fuzzy premises or rules, imprecise conclusions are drawn. However, apart from inference systems based on fuzzy rules \cite{Combs1998,Jayaram2008,Jayaram2008B}, fuzzy implication functions are also considered in other application areas like fuzzy mathematical morphology or data mining \cite{Baczynski2015}.

Partly motivated by their potential applications, the study of fuzzy implication functions has significantly grown in the last decades (see the bibliometric analysis in \cite{Laenge2021}). Indeed, some monographs \cite{Baczynski2008,Baczynski2013} and surveys \cite{Mas2007,Baczynski2008B,Baczynski2015} only devoted to the study of these operators have been published. From a theoretical perspective, the main research lines in this topic focus on the definition and study of different classes of fuzzy implication functions and the additional properties that they may satisfy. 

Although the study and proposal of additional properties is also a hot topic right now \cite{Baczynski2022C,Zhou2022B,Mis2022,Dombi2021B,Baczynski2020,Baczynski2020B,Peng2020B}, let us focus on the current state of the art regarding the research on classes of fuzzy implication functions. This research line is motivated by the fact that, depending on the context and the proper rule and its behavior, various fuzzy implication functions with different properties can be adequate \cite{Trillas2008}. The most well-known families of fuzzy implication functions are the six ones collected in the surveys \cite{Mas2007,Baczynski2008B,Baczynski2015}: $(S,N)$-implications \cite{Trillas1985}, $R$-implications \cite{Trillas1985}, $QL$-implications \cite{Mas2006}, $D$-implications \cite{Mas2006}, and Yager's $f$ and $g$-implications \cite{Yager2004}. However, many other classes of fuzzy implication functions have been defined in recent years. According to the strategy used in the definition of a certain family, we can distinguish between four classes of fuzzy implication functions:
% Different classes - millor veure survey
\begin{description}
	\item[S1.] \textbf{Classes generated from other fuzzy operators such as aggregation functions, fuzzy negations, etc.:} This strategy is based on the idea of combining adequately other fuzzy operators to obtain binary functions satisfying the axioms of the definition of a fuzzy implication function. Some of the most well-known classes such as $(S,N)$, $R$, $QL$, and $D$-implications belong to this strategy since they are generated by a t-conorm and a fuzzy negation; a t-norm; or a t-norm, a t-conorm and a fuzzy negation, respectively. More recently, other families like power-based implications \cite{Massanet2017}, Sheffer Stroke implications \cite{Baczynski2022B}, probabilistic and $S$-probabilistic implications \cite{Grzegorzewski2011}, or $(T,N)$-implications \cite{Bedregal2007} have been introduced also using this strategy.
	\item[S2.] \textbf{Classes generated from unary functions:} This strategy is based on the use of univalued functions (not necessarily fuzzy negations), often additive or multiplicative generators of other fuzzy logic connectives, to construct novel classes. These functions are usually called generators of the fuzzy implication function. This strategy experienced an important boost after Yager's $f$ and $g$-generated implications were introduced in \cite{Yager2004}.
	\item[S3.] \textbf{Classes generated from other fuzzy implication functions:} Adequately modifying the expression of already given fuzzy implication functions is another popular strategy to generate novel classes of these operators. This strategy has had an important revival lately and from the classical methods of the convex linear combination, the conjugation or the max/min construction (see \cite{Baczynski2008} for further details), more complex methods and especially, ordinal sums have recently appeared.
	\item[S4.] \textbf{Classes generated according to their final expression:} This strategy is based on fixing the desired final expression of these operators, and then studying when the corresponding functions fulfill the conditions in the definition of a fuzzy implication function. As compared with the other strategies, this one is quite new and it started in 2014, when polynomial implications were presented in \cite{Massanet2014} (see \cite{Massanet2022} for a deeper study on the polynomial implications).  
\end{description}
Apart from these four strategies, the proposal of generalizations of a certain class is quite popular, that is, to define a wider family which includes the original one. For instance, in \textbf{S1} the generalizations are usually based on considering a generalization of the fuzzy operators involved; or in \textbf{S2} they are based on weakening the conditions of the unary functions used or on generalizing the operator's expression. To express the relationship between a certain family and its generalizations, we will say that the generalizations are of the same ``type''. For example, we classify the generalizations of the $(S,N)$-implications as $(S,N)$ type implications. Having said this, intending to quantify the number of families introduced in the literature so far, we have constructed Table \ref{table:families_FI}. In this table, we have counted 146 different definitions of families of fuzzy implication functions introduced in 96 references. Due to the extensive literature on the topic, there may be other families that we have missed. However, the compilation in Table \ref{table:families_FI} is significantly broader than the corresponding one in the existing surveys \cite{Mas2007,Baczynski2008B,Baczynski2015} and monographs \cite{Baczynski2008,Baczynski2013}. Nonetheless, from Table \ref{table:families_FI} we cannot conclude that there exist 146 significantly different families of fuzzy implication functions, because these families can present intersection or even coincide. For instance, the authors in \cite{Massanet2017B} proved the equivalence of two families of fuzzy implication functions through their characterization. Until that moment, the additional properties of these two families had been studied independently. For this reason, it is of the utmost importance to study the additional properties that the operators of a certain family satisfy and to provide an axiomatic characterization of the new operators in the literature in order to find its possible relation with respect to those already known. In this respect, the characterization of several families of fuzzy implication functions have already been achieved: $(S,N)$-implications with a continuous negation \cite{Baczynski2007}, $R$-implications obtained from left-continuous t-norms \cite{Miyakoshi1985,Fodor1994}, some $QL$-implications \cite{Shi2008}, Yager’s implications \cite{Massanet2012B}, $h$-implications \cite{Massanet2012A}, probabilistic and survival $S$-implications \cite{Massanet2017B}; among others \cite{Aguilo2010,Backzynski2009,Zhou2021,Massanet2019B}. Besides, the intersections between some of the families have also been studied \cite{Baczynski2008,Baczynski2008B,Backzynski2010B}. However, the majority of families in Table \ref{table:families_FI} have not been characterized yet nor its intersection with other families has been investigated. Therefore, we can conclude that the current literature on this topic is not enough to have a proper global view of all the existing families of fuzzy implication functions.

With regard to this topic, in \cite{Massanet2019C} some guidelines for decreasing the redundancy in this field and increasing the value of those classes already introduced in the literature were pointed out: to avoid proposing new classes of fuzzy implication functions without a clear motivation; to characterize those families which have not been \linebreak \enlargethispage{\baselineskip}
\begin{landscape}
	~\newline
	\begin{table}[H]
		%\centering
		\begin{center}
			\setlength\tabcolsep{7pt}
			\renewcommand{\arraystretch}{1.75} \large
			\resizebox{20cm}{!}{
				\begin{tabular}{|cc|c|c|}
					\hline
					\multicolumn{2}{|c|}{\textbf{Strategy}}                                                                                                                                                                                                                                  & \textbf{Number} & \textbf{References} \\ \hline
					\multicolumn{1}{|c|}{\multirow{9}{*}{\textbf{\begin{tabular}[c]{@{}c@{}}Generated from other \\ fuzzy operators\\ (not fuzzy implication functions)\end{tabular}}}} & \textbf{\begin{tabular}[c]{@{}c@{}}From \\ fuzzy negations\end{tabular}}                       &         5        &     \cite{Vemuri2017,Souliotis2018,Jayaram2009,Shi2010b}                \\ \cline{2-4} 
					\multicolumn{1}{|c|}{}                                                                                                                                                  & \textbf{$(S,N)$ Type } &        13         &     \cite{Trillas1985,DeBaets1999,Yager2006,Aguilo2010B,Pradera2011,Pradera2016,VemuriJayaram2012,Dimuro2014,Li2015,Li2015B,Zhou2020,Peng2020}\\ \cline{2-4}  
					\multicolumn{1}{|c|}{}                                                                                                                                                  & \textbf{$(T,N)$ Type} & 4 &\cite{Bedregal2007,Zapata2014,Hlinena2014,Pinheiro2018}\\ \cline{2-4} 
					\multicolumn{1}{|c|}{}                                                                                                                                                  & \textbf{$R$ Type}     & 12 & \cite{Trillas1985,DeBaets1999,Yager2006,Durante2007,VemuriJayaram2012,Min2016,Liu2011,Carbonell2010,Krol2011,Ouyang2012,Liu2012B,Aguilo2013,Pereira2015}\\ \cline{2-4} 
					\multicolumn{1}{|c|}{}                                                                                                                                                  & \textbf{$QL$ Type}       & 4 & \cite{Mas2006,Mas2007B,Dimuro2017,Krol2013} \\ \cline{2-4} 
					\multicolumn{1}{|c|}{}                                                                                                                                                  & \textbf{$D$ Type}        & 3 & \cite{Mas2006,Mas2007B,Dimuro2019} \\ \cline{2-4} 
					\multicolumn{1}{|c|}{}                                                                                                                                                  & \textbf{\begin{tabular}[c]{@{}c@{}}Others derived \\ from copulas\end{tabular}}                                                              &       5          &              \cite{Grzegorzewski2011,Grzegorzewski2012,Dolati2013}       \\ \cline{2-4} 
					\multicolumn{1}{|c|}{}                                                                                                                                                  & \textbf{Power-Based}         &       1          &         \cite{Massanet2017}            \\ \cline{2-4} 
					\multicolumn{1}{|c|}{}                                                                                                                                                  & \textbf{Sheffer Stroke}                                                                        &          2       &           \cite{Baczynski2022B}          \\ \hline
					\multicolumn{1}{|c|}{\multirow{2}{*}{\textbf{\begin{tabular}[c]{@{}c@{}}Generated from unary functions \\ (not fuzzy negations)\end{tabular}}}}                         & \textbf{Yager's Type} &    23             &         \cite{Yager2004,Jayaram2006,Massanet2011A,VemuriJayaram2012,Massanet2012B,Liu2012,Massanet2013B,Liu2013,Liu2013B,XieLiu2013,ZhangZhang2017,PeiZhu2017,PeiZhu2017B,Zhou2021,Zhou2022,Xie2017}            \\ \cline{2-4} 
					\multicolumn{1}{|c|}{}                                                                                                                                                  & \textbf{Others} &      18          &        \cite{Litz1996,Smutna1999,Burillo2000,Jayaram2009,Backzynski2009B,Hlinena2012,Dombi2018,Hlinena2013B,Su2015,Vemuri2017,Baczynski2020B,Backzynski2010}             \\ \hline
					\multicolumn{1}{|c|}{\multirow{2}{*}{\textbf{\begin{tabular}[c]{@{}c@{}}Generated from other \\ fuzzy implication functions \end{tabular}}}}                         & \textbf{Ordinal Sums}              &     21            &     \cite{Mesiar2004,Su2015b,Min2016,Massanet2016b,Drygas2017,Drygas2017b,Baczynski2017,Drygas2018,Drygas2018b,Baczynski2018b,DeLima2020,FernandezSanchez2023,Wang2023}                \\ \cline{2-4} 
					\multicolumn{1}{|c|}{}                                                                                                                                                  & \textbf{Others}                                                                                &          32       &           \cite{Bandler1980,Jayaram2006,Baczynski2008,Aguilo2015,ZhangPei2017,Baczynski2015B,Aguilo2018,Massanet2018,Souliotis2019,Mesiar2020,Hallam1999,Baczynski2003,VemuriJayaram2012,Massanet2012A,Massanet2013,Reiser2013,Li2017,Yi2017,Reiser2017,Su2018,Mesiar2019}          \\ \hline
					\multicolumn{2}{|c|}{\textbf{According to their final expression}}                                                                                                                                                                                                       &        3         &      \Cite{Massanet2014,Massanet2015,Massanet2016}               \\ \hline
					\multicolumn{2}{|c|}{\textbf{Total}}                                                                                                                                                                                                       &      146           &           96          \\ \hline
			\end{tabular}}
		\end{center}
		\caption{Classification of several families of fuzzy implication functions according to their construction methods. Each reference corresponds to the papers in which the corresponding families were introduced.}\label{table:families_FI}
	\end{table}
\end{landscape}

\noindent characterized yet; to solve some important open problems in the literature; and to open new application fields for fuzzy implication functions. In this monograph, we have precisely followed these guidelines.

Consequently, none of our main goals corresponds to the introduction of new classes of fuzzy implication functions or additional properties. Indeed, in general terms our principal objectives are: providing the characterization of two existing families in the literature, being one of them a well-known open problem; studying the family of fuzzy implication functions characterized by the fact that they satisfy a valuable property for approximate reasoning; and exploring the potential of fuzzy implication functions in a new application area within knowledge discovery. This is not to say that, due to the requirements of the particular problem, we do not define new families or additional properties. However, we do so with our sights focused on the main objective.

\section{Objectives, thesis structure and research contributions}

This thesis has four well-defined objectives, which have been addressed in four separate chapters. In the introduction of each corresponding chapter, a detailed contextualization of the problem is given. Therefore, in this section we only give a brief summary of the contributions linked to each objective:

\begin{description}
	\item[Objective 1.] \textit{The characterization of generalized $(h,e)$-implications.}\\
	In Chapter \ref{chapter:heimplications} the open problem of the characterization of generalized $(h,e)$-im\-pli\-ca\-tions \cite{Massanet2011A} is studied and totally solved. First of all, we study when this family fulfills some of the main additional properties of fuzzy implication functions and we obtain a representation theorem that describes the structure of a generalized $(h,e)$-implications in terms of two families of fuzzy implication functions via the horizontal threshold construction method. These two families can be interpreted as particular cases of the $(f,g)$ and $(g,f)$-implications, which are two families of fuzzy implication functions that generalize the well-known $f$ and $g$-generated implications proposed by Yager \cite{Yager2004} through a generalization of the internal factors $x$ and $\frac{1}{x}$, respectively. The additional properties of these two families are also studied in detail and the intersection between them is characterized. Subsequently, we provide the characterization of the two subfamilies of $(f,g)$ and $(g,f)$-implications that are related to the structure of generalized $(h,e)$-implications, called $(f,e)$ and $(g,e)$-implications. From these two characterizations and the representation theorem, we derive an axiomatic characterization of generalized $(h,e)$-implications. The three characterizations presented rely on two new additional properties of fuzzy implication functions which are modifications of the law of importation.
	
	\item[Objective 2.] \textit{The study of the family of fuzzy implication functions characterized by the fact that they satisfy the invariance property with respect to the positive powers of a continuous Archimedean t-norm.}
	
	The invariance property with respect to the positive powers of continuous t-norms has been recently introduced as an additional property of fuzzy implication functions which is particularly interesting in the area of approximate reasoning \cite{Massanet2017}. Although in this same paper the authors proposed power-based implications as a class of fuzzy implication functions that fulfills the invariance property with respect to many continuous t-norms, it is also pointed out that this family does not satisfy many of the most well-known additional properties of fuzzy implication functions. In Chapter \ref{chapter:tpowerinvariant} we study the invariance property in a more general way than the existing perspectives in the literature with the intention of obtaining a more versatile class of fuzzy implication functions that also satisfies this property. First, we provide the characterization of all binary functions which are invariant with respect to the positive powers of a continuous Archimedean t-norm. From this result, we define the families of strict/nilpotent $T$-power invariant implications as those classes which include all fuzzy implication functions that satisfy the invariance property with respect to a certain strict/nilpotent t-norm~$T$. We study when the members of these families do satisfy other additional properties apart from the invariance. From this study, it is proved that there are fuzzy implication functions from these two families satisfying important properties such as the exchange principle or the generalized modus ponens, among others. This analysis leads to the characterization of the intersection of these families with some of the most usual classes of fuzzy implication functions.    
	
	
	\item [Objective 3.] \textit{The characterization of $(S,N)$-implications with a non-continuous fuzzy negation.}
	
	Although the characterization of $(S,N)$-implications when $N$ is a continuous negation was presented in 2007 \cite{Baczynski2007}, the characterization in the case when $N$ is a non-continuous negation has remained one of the most significant open problems in the study of fuzzy implication functions for the last decades \cite{Baczynski2008,Baczynski2015}. In Chapter \ref{chapter:chsnimplications} we deeply analyze this problem and we provide new significant advances. Since this objective has been more laborious to study, we split our contributions in three different parts:
	\begin{itemize}
		\item[$\star$] \textit{A first characterization of $(S,N)$-implications when $N$ is a non-continuous negation. Proof of the equivalence between the characterization of $(S,N)$-implications and the problem of the completion of t-conorms.}
		
		In Section \ref{section:characterization1}  we present a general characterization of $(S,N)$-implications. From this result, we conclude that the characterization of $(S,N)$-implications where $N$ is a non-continuous fuzzy negation is equivalent to the problem of the completion of a t-conorm whose expression is unknown in a subregion of the unit square, where this subregion is determined by the discontinuities of the respective fuzzy negation. Nonetheless, we point out that to determine when a t-conorm can be completed is far from being an easy condition to verify. Thus, with the aim of providing another characterization of $(S,N)$-implications based on the explicit construction of the t-conorm $S$ we focus on the completion problem. At this point, by the duality between t-norms and t-conorms, we prove that our problem is equivalent to the study of the completions of t-norms. Thus, to be coherent with the literature on this topic \cite{Alsina2006,Klement2000} we study this dual problem.

		\item[$\star$] \textit{The characterization and construction of the continuous completions of some pre-t-norms.}
		
		The question of whether a continuous t-norm whose values in a subregion of the unit square are unknown can be (uniquely) completed is a classical and significant problem in the study of these operators \cite{Alsina2006,Klement2000}. The results regarding this topic are valuable since they disclose important information about their structure, for instance, which subregions of the domain determine the rest of the values uniquely. According to the previous item, in this report we have introduced a new motivation for the study of this well-known problem.
		
		In Section \ref{subsec:related_bibliography}, we present an overview of the results in the literature regarding the completions of t-norms. From this study we conclude that the completion problem linked to the characterization of $(S,N)$-implications is too complex to be approached in a general way. Thus, as a first step we focus on the regions derived from the case when $N$ has only one point of discontinuity and $S$ is a continuous t-conorm. In this particular case, the interest relies on the determination of the continuous completions of pre-t-norms defined on eight specific regions which, up to our knowledge, have not been previously considered in the literature. In Sections \ref{subsection:cancellative_completions} and \ref{subsection:completions_conditionalcase} we provide the continuous completions of cancellative and conditionally cancellative pre-t-norms defined on these eight regions.
		
		The obtained results are very different depending on the region and the cancellative and the conditionally cancellative situations, so several cases have had to be analyzed and a specific approach was necessary for almost each case. Depending on the case, the corresponding pre-t-norm can be completed uniquely or it has an infinite number of completions, but in all cases we provide the construction of all the continuous completions in terms of an additive generator.
		
		\item[$\star$] \textit{A second characterization of $(S,N)$-implications when $N$ has one point of discontinuity.}
		
		In Section \ref{section:characterization2} we gather the results exposed in the two above items to provide a new characterization of $(S,N)$-implications in the case when $N$ has one point of discontinuity and $S$ is the maximum or a continuous Archimedean t-conorm. In this new characterization, all the possible representations of the corresponding binary function as an $(S,N)$-implication are constructed.
		
		
	\end{itemize}
	\item[Objective 4.] \textit{The proposal of a new perspective of subgroup discovery based on fuzzy implication functions.}
	
	Subgroup discovery (SD) is a widely known descriptive data mining technique designed for identifying subgroups of data which are interesting with respect to a target variable \cite{Atzmueller2015,Helal2016}. The importance of a subgroup is numerically quantified by a quality measure, which is selected according to the objectives of the task at hand. Each subgroup is normally represented in the form of a rule $\text{Condition} \Rightarrow \text{Target}$, where ``Target'' is the property of interest and ``Condition'' is a conjunction of features. 
	One of the key aspects in SD is the interpretability of the results, so the output should be simple enough to be understood and analyzed by an expert. This requirement makes natural to consider the use of linguistic fuzzy rules to model subgroups. In accordance, several SD algorithms based on fuzzy logic have already been proposed in the literature \cite{Herrera2011}. However, up to our knowledge, these algorithms are only valid for categorical target variables and the rule form in the definition of a subgroup is interpreted as co-occurrence rather than a logical conditional. In this way, we propose a new approach that solves these two disadvantages by introducing the use of fuzzy implication functions. 
	Our contribution in Chapter \ref{chapter:sd} is to design some SD algorithms based on linguistic fuzzy rules modeled by fuzzy implication functions. Due to the structure of these operators, the corresponding subgroups can be interpreted as conditional statements and the numeric target can be modeled as a fuzzy linguistic variable. In our study, we adapt and reinterpret several SD quality measures for this new framework and we test and analyze the adequacy of the different fuzzy operators involved. 
\end{description}

The structure of this report is in line with the objectives' presentation: Chapter \ref{chapter:preliminaries} is devoted to the preliminaries, in which we introduce all the definitions and results that are necessary to understand the contributions presented in this monograph; in Chapters \ref{chapter:heimplications}, \ref{chapter:tpowerinvariant}, \ref{chapter:chsnimplications} and \ref{chapter:sd} we discuss Objectives 1, 2, 3 and 4, respectively. The report ends in Chapter \ref{chapter:conclusions} with some conclusions and future work.

