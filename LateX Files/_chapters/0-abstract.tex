\chapter*{Abstract}
\addcontentsline{toc}{chapter}{Abstract}

Fuzzy implication functions are one of the fundamental operators of fuzzy logic, in which they generalize the concept of the classical implication from the set $\{0,1\}$ to the unit interval $[0,1]$. The important role that these connectives play both in theory and applications has led them to become one of the most relevant research areas within fuzzy logic. In this thesis, we have mainly focused on the study and resolution of some open problems regarding the characterizations and intersections of different families.

The contents of this monograph are boldly separated into four objectives, which in turn have resulted in various contributions to the field.

To begin with, we present a characterization of generalized $(h,e)$-im\-pli\-ca\-tions. This result is obtained by first providing a representation theorem based on the horizontal threshold method that describes the structure of these operators in terms of two families which are generalizations of Yager's implications. Thus, by finding the characterizations of these two families we have transformed the representation theorem to an axiomatic characterization of generalized $(h,e)$-implications in terms of their own properties.

Second, we characterize the families of fuzzy implication functions which are invariant with respect to the positive powers of a strict/nilpotent t-norm. Further, we thoroughly study which additional properties apart from the invariance are fulfilled by these two families and we disclose that their structure stand out with respect to the most well-known families by studying the corresponding intersections.

Third, we provide significant advances on the renowned open problem of the characterization of $(S,N)$-implications when $N$ is a non-continuous fuzzy negation. We first prove that the problem is equivalent to the completion of t-conorms whose expression is unknown in a region which is determined by the discontinuities of $N$. Accordingly, we present new results on the dual problem of the completions of t-norms from which we derive a second axiomatic characterization of $(S,N)$-implications in some particular cases.

Finally, we propose a novel framework for the subgroup discovery data mining technique based on the use of fuzzy implication functions for modeling subgroups as fuzzy rules. We thoroughly describe this new setting and we study which properties should be imposed on the involved fuzzy operators. Further, we design and implement some subgroup discovery algorithms and we show that our perspective provides valuable knowledge which is different from other existing approaches.

\chapter*{Resumen}
\addcontentsline{toc}{chapter}{Resumen}

Las funciones de implicación borrosas son uno de los operadores primordiales de la lógica borrosa, en la cual generalizan el concepto de la implicación clásica del conjunto $\{0,1\}$ al intervalo unidad $[0,1]$. El importante papel que estos conectivos tienen tanto en la teoría como en las aplicaciones los ha llevado a convertirse en uno de los campos de investigación más relevantes dentro de la lógica borrosa. En esta tesis, nos hemos enfocado principalmente en el estudio y resolución de algunos problemas abiertos en relación a las caracterizaciones e intersecciones de diferentes familias.

Los contenidos de esta monografía están claramente separados en cuatro objetivos, que a su vez han resultado en varias contribuciones a este campo.

Para empezar, presentamos la caracterización de las $(h,e)$-implicaciones ge\-ne\-ra\-li\-za\-das. Este resultado se obtiene primero proporcionando un teorema de representación basado en el método del umbral horizontal, que describe la estructura de estos operadores en términos de dos familias que son generalizaciones de las implicaciones de Yager. Por consiguiente, a partir de las caracterizaciones de estas dos familias se ha transformado el teorema de representación en una caracterización axiomática de las $(h,e)$-implicaciones generalizadas con base en sus propias propiedades. 

En segundo lugar, caracterizamos las familias de funciones de implicación borrosas que son invariantes respecto de las potencias positivas de una t-norma estricta/nilpotente. Además, estudiamos en detalle qué propiedades adicionales aparte de la invariancia satisfacen estas dos familias y, a partir del estudio de las intersecciones respectivas, revelamos que su estructura destaca en relación a la de las familias más conocidas.

En tercer lugar, aportamos avances significativos al renombrado problema abierto de la caracterización de las $(S,N)$-implicaciones cuando $N$ es una negación borrosa no continua. Primero demostramos que el problema es equivalente a la completación de t-conormas cuya expresión es desconocida en una región que depende de las discontinuidades de $N$. En consecuencia, presentamos nuevos resultados del pro\-ble\-ma dual de las completaciones de t-normas, de los cuales derivamos una segunda caracterización axiomática de las $(S,N)$-implicaciones en algunos casos particulares.

Finalmente, proponemos un nuevo marco para la técnica de minería de datos de descubrimiento de subgrupos basada en el uso de funciones de implicación borrosas para modelar subgrupos como reglas borrosas. Describimos de forma detallada esta nueva configuración y estudiamos qué propiedades deberían ser impuestas en los operadores borrosos involucrados. Además, diseñamos e implementamos algunos algoritmos de descubrimiento de subgrupos y mostramos que nuestra perspectiva provee conocimiento interesante que es diferente al de otros métodos existentes.

\chapter*{Resum}
\addcontentsline{toc}{chapter}{Resum}

Les funcions d'implicació borroses són un dels operadors fonamentals de la lògica borrosa, en la qual generalitzen el concepte de la implicació clàssica del conjunt $\{0,1\}$ a l'interval unitat $[0,1]$. L'important paper que aquests connectius tenen tant a la teoria com en les aplicacions els ha duit a convertir-se en un els camps d'investigació més rellevants dins de la lògica borrosa. En aquesta tesi, ens hem enfocat principalment en l'estudi i resolució d'alguns problemes oberts en relació amb les caracteritzacions i interseccions de diferents famílies.

Els continguts d'aquesta monografia estan clarament separats en quatre objectius, que al seu torn han resultat en diverses contribucions en aquest camp.

Per començar, presentam la caracterització de les $(h,e)$-implicacions generalitzades. Aquest resultat s'obté primer proporcionant un teorema de representació basat en el mètode del llindar horitzontal, que descriu l'estructura d'aquests operadors en termes de dues famílies que són generalitzacions de les implicacions de Yager. Per consegüent, a partir de les caracteritzacions d'aquestes dues famílies s'ha transformat el teorema de representació en una caracterització axiomàtica de les $(h,e)$-implicacions generalitzades amb base en les seves pròpies propietats.


En segon lloc, caracteritzam les famílies de les funcions d'implicació borroses que són invariants respecte de les potències positives d'una t-norma estricta/nilpotent. A més, estudiam a fons quines propietats addicionals, a part de la invariància, es compleixen per aquestes dues famílies i, mitjançant l'estudi de les interseccions corresponents, revelam que la seva estructura destaca respecte de les famílies més conegudes.

En tercer lloc, proporcionam avanços significatius en el famós problema obert de la caracterització de les $(S,N)$-implicacions quan $N$ és una negació borrosa no contínua. Primer demostram que el problema és equivalent a la completació de t-conormes l'expressió de la qual és desconeguda en una regió que està determinada per les discontinuïtats de $N$. En conseqüència, presentam nous resultats sobre el problema dual de les completacions de t-normes dels quals obtenim una segona caracterització axiomàtica de les $(S,N)$-implicacions en alguns casos particulars.

Finalment, proposam un nou marc per a la tècnica de mineria de dades de descobriment de subgrups basat en l'ús de funcions d'implicació borroses per al modelatge de subgrups com a regles borroses. Descrivim a fons aquesta nova configuració i estudiam quines propietats haurien d'imposar-se als operadors borrosos implicats. A més, dissenyam i implementam alguns algorismes de descobriment de subgrups i mostram que la nostra perspectiva proporciona coneixement interessant que és diferent d'altres enfocaments existents.


% All three characterizations are based on two novel properties which are modifications of the law of importation.